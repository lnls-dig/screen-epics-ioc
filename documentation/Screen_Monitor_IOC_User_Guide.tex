\documentclass[openany]{article}
\usepackage[a4paper,margin=1in,bottom=1.5in]{geometry} % define margins. Bottom margin is used to lift a little bit the page number.
\usepackage[english]{babel} % document language is english
\usepackage{tikz} % for drawing (currently not used).
\usepackage{graphicx} % for including images
\usepackage[export]{adjustbox}
\usepackage{fancyhdr} % used for creating headers and footers. only used in title page in this document.
\usepackage{tabularx} % creation of more complex tables
\usepackage{longtable} % tables can span multiple pages
\usepackage{array} % allow elements of tabular environment to have vertical alignment, e.g., center alignment.
\usepackage{nameref} % make it possible to reference by name
\usepackage{hyperref} % allow hiperlinks (links to other document parts and extern links)
\usepackage{etoc} % used for generation of section local table of contents
\usepackage{placeins} % defines the \FloatBarrier command
\usepackage{xcolor} % for modifying box color
\usepackage{adjustbox} % for modifying box parameters
\usepackage{textcomp} % for having the degree symbol
\usepackage{listings} % for code snippets

% configure a listing environment for code snippets
\lstnewenvironment{code}
    {\lstset{linewidth=\linewidth,
             backgroundcolor=\color{white},
             frame=single,basicstyle=\normalsize\ttfamily\color{black},
             breaklines=true
             }}
    {}

% Define graphics path
\graphicspath{{figs/}}

% Configure the cross reference hyper links color
\hypersetup{
    colorlinks=true,
    linkcolor=blue,
}

\newcolumntype{C}{>{\centering\arraybackslash}X} % new column type for tabularx
                         % centered (\centering), adjust width in order to fill table width (X type)

% Configure header in 'titlepage'
\pagestyle{fancy}
\lhead{\includegraphics[width=4.5cm]{logo_cnpem}}
\rhead{\includegraphics[width=4cm]{logo_lnls}}
\renewcommand{\headrulewidth}{0pt}
\setlength{\headheight}{52pt}
% Clean footer
\fancyfoot{}

% increase table height factor a little bit (taller cells)
\renewcommand{\arraystretch}{1.5}

%==== Begin DOCUMENT ====
\begin{document}

%--- Begin title page ---
\begin{titlepage}

% Add header to this page
\thispagestyle{fancy}

% Center elements
\begin{center}

% title of title page
\topskip0pt % perfectly centered
\vspace*{\fill}
\textbf{\Huge Screen Monitor EPICS IOC User Guide}\\[20pt]
\textbf{\Huge Version 1.0}\\[20pt]
\textbf{\Huge May/2020}
\vspace*{\fill}

% footer of title page
\vfill
\textbf{Beam Diagnostics Group (DIGS)}\\[5pt]
\textbf{Brazilian Synchrotron Light Laboratory (LNLS)}\\[5pt]
\textbf{Brazilian Center for Research in Energy and Materials (CNPEM)}
\end{center}

\end{titlepage}
%--- End of title page ---

\newpage
\pagestyle{plain} % restore default page style

%--- About this manual ---
\paragraph{}{\Large\bfseries About this manual}

\paragraph{} This manual provides an overview of the Screen Monitor EPICS Soft IOC. It is assumed that the reader is familiar with the basics of EPICS.

%--- Table of contents ---
\tableofcontents

\newpage
%--- Section: Introduction ---
\section{Introduction}

\paragraph{} The Screen Monitor EPICS IOC is a Soft IOC used as a high level layer for controlling the Fluorescent Screen Monitors in the Sirius accelerator (not beamlines). The IOC tries to provide intuitive control of screen position and LED. It communicates to hardware by directly linking its records to PVs of the motion controller IOC. The controller used in this application at Sirius is the Galil DMC 30017, and the IOC support used for it can be found at \url{https://github.com/lnls-dig/galil-dmc30017-epics-ioc}.

\paragraph{} The Screen Monitor IOC repository is \url{https://github.com/lnls-dig/screen-epics-ioc}.

%--- Section: Installation ---
\section{Installation}

    \paragraph{Shortcut} It is possible to run a \emph{docker image} containing the IOC and all of its dependencies. Go to the \hyperref[sec:run-with-docker]{Running the IOC from a docker container} section to see how to do it.

    \subsection{Installing without docker}

        \paragraph{Before everything} Make sure you have EPICS Base 3.15 and the following EPICS modules installed (older versions might work but are not recommended):

        \begin{itemize}
          \item autosave R5-9
          \item seq 2-2-6
          \item calc R3-7
          \item busy R1-7
        \end{itemize} 

        \paragraph{} Clone the IOC repository and checkout the desired commit (in case you are not using master).

            \vspace{1mm}
            \begin{code}
git clone https://github.com/lnls-dig/screen-epics-ioc.git <install location>
cd <install location>
git checkout <commit>
            \end{code}
            \vspace{1mm}

        \paragraph{} Create a \emph{RELEASE.local} file inside the \emph{configure/} directory and add the paths to EPICS Base and external modules. An example file is available at \emph{configure/RELEASE.local.example}.

            \vspace{1mm}
            \begin{code}
echo 'EPICS_BASE=<EPICS Base path>' > configure/RELEASE.local
echo 'AUTOSAVE=<Autosave path>' >> configure/RELEASE.local
echo 'SNCSEQ=<seq path>' >> configure/RELEASE.local
echo 'CALC=<calc path>' >> configure/RELEASE.local
echo 'BUSY=<busy path>' >> configure/RELEASE.local
            \end{code}
            \vspace{1mm}

        \paragraph{} From the IOC $<$TOP$>$ directory, run make:

            \vspace{1mm}
            \begin{code}
make
            \end{code}
            \vspace{1mm}

%--- Section: Running the IOC without docker ---
\section{Running the IOC without docker}

    After successfully building the IOC, from the IOC top directory, run:

        \vspace{1mm}
        \begin{code}
cd iocBoot/iocScreen &&
./runScreen.sh <options>
        \end{code}
        \vspace{1mm}

    The available options can be listed by passing any invalid option to the startup script, such as:

        \vspace{1mm}
        \begin{code}
./runDiffCtrl.sh -h
        \end{code}
        \vspace{1mm}

    or can be consulted in the README.md file at the IOC $<$TOP$>$ directory or project page \url{https://github.com/lnls-dig/screen-epics-ioc}.

%--- Section: Running the IOC from a docker container ---
\section{Running the IOC from a docker container}\label{sec:run-with-docker}

    \paragraph{} Make sure you have docker installed and that you understand the basics of creating, running and acessing a container, and downloading an image.

    \subsection{Downloading the image (IOC + dependencies)}

        \paragraph{} The image named \textbf{lnlsdig/screen-epics-ioc} contains the compiled IOC and dependencies, and can be downloaded to your machine by running:

        \vspace{1mm}
        \begin{code}
docker pull lnlsdig/screen-epics-ioc:<IOC_TAG>
        \end{code}
        \vspace{1mm}

        where \emph{$<$IOC\_TAG$>$} should be replaced by the desired IOC version.

        \paragraph{} The IOC version list can be found at \url{https://hub.docker.com/r/lnlsdig/screen-epics-ioc/tags}.

        After the IOC image is downloaded, you can instante it by running a container that uses it.

    \subsection{Running a docker container with that image}

        \paragraph{} One way to run the container is to do:

        \vspace{1mm}
        \begin{code}
docker run -it --network host --restart always --name <CONTAINER_NAME> lnlsdig/screen-epics-ioc:<IOC_TAG> -P <PREFIX1> -R <PREFIX2> <other options>
        \end{code}
        \vspace{1mm}

        Which is going to run the container starting at its default entrypoint, that is, the IOC startup script. The parameter $<$IOC\_TAG$>$ should be replaced by the IOC version. All the arguments after it are passed to the IOC startup script when run in this way. The $<$PREFIX1$>$ and $<$PREFIX2$>$ are the IOC EPICS PVs prefix, combined as follows: \emph{PV\_PREFIX = PREFIX1 + PREFIX2}. The other startup script arguments should be provided as well. They are listed in the IOC README.md file.

        \paragraph{} \textbf{If you run this command before downloading the image, the image is automatically downloaded, provided there is internet connection}.

        \paragraph{} Below you find an explanation for each flag we used in the above command.

        \begin{itemize}
          \item -it: interactive (STDIN open) and allocate pseudo-TTY. Since we did not change the container entrypoint, your terminal is going to connect to the IOC shell.
          \item --network host: configures the container to use the host's network stack inside the container. It is the simplest configuration to use, although it provides no isolation.
          \item --restart always: restarts the container automatically when it exits.
          \item --name $<$CONTAINER\_NAME$>$: configures the container name as $<$CONTAINER\_NAME$>$.
        \end{itemize}

        \paragraph{} In order to stop the docker container, i.e., the IOC, run:

        \vspace{1mm}
        \begin{code}
docker stop <CONTAINER_NAME>
        \end{code}
        \vspace{1mm}

        \paragraph{} If you want to delete the container, after stopping it, run:

        \vspace{1mm}
        \begin{code}
docker rm <CONTAINER_NAME>
        \end{code}
        \vspace{1mm}

        \paragraph{} You can also run the contaier with the \emph{--rm} flag to clean up the container and remove the file system when it exits.

%--- Section: Screen Monitor Camera Integration ---
\section{Screen Monitor Camera Integration}\label{sec:dev-reference-frame}

    \paragraph{} The Screen Monitor High Level IOC provides an abstraction layer for simple control of its screen position by sending commands to the controller low-level IOC. On the other hand, although the Screen OPI provides image data and PVs to control the camera, the camera control and aquisition is performed only by the \emph{Camera Low-level IOC}. The reason for it is that it would be inefficient to transmit the image data through channel access to the high-level IOC in order to make it available to the user. Since the data size can be relatively large, bandwidth (in the case where the low and high level IOCs are not running on the same machine) and memory would be wasted.

    \paragraph{} The solution chosen for this application was to use \emph{high level aliases} in the low-level camera IOC and reference these aliases in the Screen Monitor OPI. For instance, if the prefix used for the PVs in the camera IOC was \emph{cam:} and the Screen Monitor IOC used the prefix \emph{scrn:}, then the PV \emph{cam:Data-Mon} would have an alias referenced at the Screen OPI named \emph{scrn:ImgData-Mon}.

    \paragraph{} If the prefix used for the aliases in the low-level camera IOC is the same used in the high-level Screen Monitor IOC, then from the user perspective, the image PVs are part of the Screen device.

    \begin{figure}[!h]
        \caption{Screen Monitor IOC and low-level Camera IOC}
        \label{fig:cam-integ}
        \centering
        \includegraphics[width=1.0\textwidth]{screen_ioc_cam_integration}
    \end{figure}
    \FloatBarrier

%--- Section: Screen Selection ---
\section{Screen Selection}

    \begin{figure}[!h]
        \caption{Screen type selection}
        \label{fig:opi-screen-type}
        \centering
        \includegraphics[width=1.0\textwidth]{screen_opi_screen_type}
    \end{figure}
    \FloatBarrier

    \paragraph{} In order to move a given screen type in front of the beam, just select the desired option on the button indicated by the red rectangle in the example image. While the screen is moving, the \emph{Done Moving} LED will be off and the status text is going to display \emph{Unknown}, indicating that the motor is not yet in one of the known screen positions. When the \emph{Done Moving} LED turns on, the motor has finished moving. The status text should display the correct desired screen position.

%--- Section: Calibration of Screen Positions ---
\section{Calibration of Screen Positions}

    \begin{figure}[!h]
        \caption{Screen positions calibration}
        \label{fig:opi-screen-cal}
        \centering
        \includegraphics[width=1.0\textwidth]{screen_opi_screen_cal}
    \end{figure}
    \FloatBarrier

    \paragraph{} The red rectangle in the example figure shows the buttons and boxes used for screen position calibration. The boxes at the left should contain the position, in motion controller engineering units, of each screen type. When one of the positions is selected, the motor is commanded to the corresponding configured position. The arrow buttons at the right of each box can be used to transfer the current encoder reading to the corresponding box. This is appropriate when configuring the screen position based on some visual feedback. In this case, the motor can be moved until reaching a satisfactory position and the encoder value transfered to one of the calibration position values. The motor can be moved to any position through the \emph{Abs Position} set point. This widget is actually linked to a low-level PV and only works when the motion controller IOC is running. The \emph{Err Tolerance} value defines the tolerance used by the screen type status. In order for the status to indicate a given screen type, the motor must be positioned in the configured value with a maximum error given by the tolerance.

%--- Section: LED Control and Calibration ---
\section{LED Control and Calibration}

    \begin{figure}[!h]
        \caption{LED control and calibration}
        \label{fig:opi-led-cal}
        \centering
        \includegraphics[width=1.0\textwidth]{screen_opi_led_cal}
    \end{figure}
    \FloatBarrier

    \paragraph{} For Screen Monitors that use LEDs, the \emph{Enable LED} button controls whether any voltage is provided to the LED terminals. If the LED is enabled, then the \emph{LED Power} specifies what percentage of maximum brightness the LED should have. The \emph{Threshold} specifies the minimum voltage that the LEDs should receive when enabled, while the \emph{Scale Factor} specifies the translation between LED Power and applied output voltage though the formula: $A.e^{B.C}$, where A = LED voltage threshold, B = LED power level (0 to 100), and C = LED power scale factor.

\newpage
\section{Process Variables Description}\label{sec:process-variables}

    Each IOC instance should add a prefix to the process variables indicating which device it controls.

    % Process Variables description table
    \begin{longtable}{| m{4.5cm} m{2.5cm}  m{7.0cm} |}
        \caption{Application Process Variables} \\ \hline
        \bfseries Name & \bfseries Data Type & \bfseries Description \label{tab:PV-description} \endfirsthead
        \caption{Application Process Variables} \\ \hline
        \bfseries Name & \bfseries Data Type & \bfseries Description \endhead \hline
        % --- row ---
        PV\_NAME & DATA\_TYPE & \begin{tabular}{@{}m{6cm}@{}}
                            DESCRIPTION
            \end{tabular} \hypertarget{pv:mtr-ctrl-prefix-cte}{}\\ \hline
        % --- row ---
        MtrCtrlPrefix-Cte & String (char[40]) & \begin{tabular}{@{}m{6cm}@{}}
                Prefix used by motion controller IOC PVs.
            \end{tabular} \hypertarget{pv:cam-prefix-cte}{}\\ \hline
        % --- row ---
        CamPrefix-Cte & String (char[40]) & \begin{tabular}{@{}m{6cm}@{}}
                Prefix used by camera IOC PVs.
            \end{tabular} \hypertarget{pv:rst-cmd}{}\\ \hline
        % --- row ---
        Rst-Cmd & Enum: No (0), Yes (1) & \begin{tabular}{@{}m{6cm}@{}}
                Reset command. When set to 1, this PV sends a reset command to both motor controller and camera low level IOCs. The reset status is showed by \hyperlink{pv:rst-done-mon}{RstDone-Mon} PV.
            \end{tabular} \hypertarget{pv:rst-done-mon}{}\\ \hline
        % --- row ---
        RstDone-Mon & Enum: In Progress (0), Finished (1) & \begin{tabular}{@{}m{6cm}@{}}
                Reset done status. This PV shows a combination of the reset status of both motor controller and camera low level IOCs. When both low level IOCs have finished resetting, this PV is set to \emph{Finished} (1), otherwise its status remains \emph{In Progress} (0).
            \end{tabular} \hypertarget{pv:home-cmd}{}\\ \hline
        % --- row ---
        Home-Cmd & Enum: No (0), Yes (1) & \begin{tabular}{@{}m{6cm}@{}}
                Homing command. When set to 1, this PV sends a homing command to the motor low level IOC. The motor then moves back (according to what 'back' means for its direction of movement) until it hits the reverse limit switch. After hitting the switch, the incremental encoder position is reset to zero. The homing status is showed by \hyperlink{pv:home-done-mon}{HomeDone-Mon} PV.
            \end{tabular} \hypertarget{pv:home-done-mon}{}\\ \hline
        % --- row ---
        HomeDone-Mon & Enum: In Progress (0), Finished (1), Error (2) & \begin{tabular}{@{}m{6cm}@{}}
                Homing operation status. This PV indicates the status \emph{Finished} when the homing operationg finishes successfully. When homing fails, i.e., the motor stops before hitting the home position, the PV displays the \emph{Error} status.
            \end{tabular} \hypertarget{pv:fluor-scrn-pos}{}\\ \hline
        % --- row ---
        FluorScrnPos-SP & Float: min -1e3, max 1e3 & \begin{tabular}{@{}m{6cm}@{}}
                Fluorescent screen position. This is the set point position sent to the controller when the \emph{Fluorescent} screen type is selected.
            \end{tabular} \hypertarget{}{}\\ \hline
        % --- row ---
        FluorScrnPos-RB & Float & \begin{tabular}{@{}m{6cm}@{}}
                The readback value of the \emph{Fluorescent} screen position.
            \end{tabular} \hypertarget{pv:get-fluor-scrn-pos-cmd}{}\\ \hline
        % --- row ---
        GetFluorScrnPos-Cmd & Enum: Off (0), On (1) & \begin{tabular}{@{}m{6cm}@{}}
                Command to transfer the current encoder reading to the \emph{Fluorescent} screen position (\emph{FluorScrnPos-SP}). This is useful for avoiding typing errors when specifying the screen associated position. The command is only effective when it receives a value $\neq 0$.
            \end{tabular} \hypertarget{pv:cal-scrn-pos}{}\\ \hline
        % --- row ---
        CalScrnPos-SP & Float: min -1e3, max 1e3 & \begin{tabular}{@{}m{6cm}@{}}
                Calibration screen position. This is the set point position sent to the controller when the \emph{Calibration} screen type is selected.
            \end{tabular} \hypertarget{}{}\\ \hline
        % --- row ---
        CalScrnPos-RB & Float & \begin{tabular}{@{}m{6cm}@{}}
                The readback value of the \emph{Calibration} screen position.
            \end{tabular} \hypertarget{pv:get-cal-scrn-pos-cmd}{}\\ \hline
        % --- row ---
        GetCalScrnPos-Cmd & Enum: Off (0), On (1) & \begin{tabular}{@{}m{6cm}@{}}
                Command to transfer the current encoder reading to the \emph{Calibration} screen position (\emph{CalScrnPos-SP}). This is useful for avoiding typing errors when specifying the screen associated position. The command is only effective when it receives a value $\neq 0$.
            \end{tabular} \hypertarget{pv:none-scrn-pos}{}\\ \hline
        % --- row ---
        NoneScrnPos-SP & Float: min -1e3, max 1e3 & \begin{tabular}{@{}m{6cm}@{}}
                None (or Out) screen position. This is the set point position sent to the controller when the \emph{None} screen type is selected.
            \end{tabular} \hypertarget{}{}\\ \hline
        % --- row ---
        NoneScrnPos-RB & Float & \begin{tabular}{@{}m{6cm}@{}}
                The readback value of the \emph{None} screen position.
            \end{tabular} \hypertarget{pv:get-none-scrn-pos-cmd}{}\\ \hline
        % --- row ---
        GetNoneScrnPos-Cmd & Enum: Off (0), On (1) & \begin{tabular}{@{}m{6cm}@{}}
                Command to transfer the current encoder reading to the \emph{None} screen position (\emph{NoneScrnPos-SP}). This is useful for avoiding typing errors when specifying the screen associated position. The command is only effective when it receives a value $\neq 0$.
            \end{tabular} \hypertarget{pv:scrn-type}{}\\ \hline
        % --- row ---
        ScrnType-Sel & Enum: None (0), Calibration (1), Fluorescent (2) & \begin{tabular}{@{}m{6cm}@{}}
                Moves the selected screen type in the beam path. For the \emph{None} screen option, this means clearing the beam path.
            \end{tabular} \hypertarget{}{}\\ \hline
        % --- row ---
        ScrnType-Sts & Enum: None (0), Calibration (1), Fluorescent (2), Unknown (3) & \begin{tabular}{@{}m{6cm}@{}}
                The status showing which screen is currently positioned in the beam path. The status shows the screen type associated with the current encoder reading, given a tolerance margin provided by \emph{AcceptedErr-RB}. When no screen is associated with the current reading, then the status is \emph{Unknown}.
            \end{tabular} \hypertarget{pv:accepted-err}{}\\ \hline
        % --- row ---
        AcceptedErr-SP & Float: min -1e3, max 1e3 & \begin{tabular}{@{}m{6cm}@{}}
                The tolerance margin, in encoder engineering units, considered when indicating which screen type is positioned in the beam path, given the configured screen positions. A tolerance is necessary, since not all decimal digits provided by the encoder might be significant to the application. 
            \end{tabular} \hypertarget{}{}\\ \hline
        % --- row ---
        AcceptedErr-RB & Float & \begin{tabular}{@{}m{6cm}@{}}
                The readback value of the tolerance margin associated to the expected value for the encoder reading at a given screen position, in encoder engineering units.
            \end{tabular} \hypertarget{pv:done-mov-mon}{}\\ \hline
        % --- row ---
        DoneMov-Mon & Enum: No (0), Yes (1) & \begin{tabular}{@{}m{6cm}@{}}
                Indicates when the motor has completely finished moving to a given position. The move is also considered to be finished if it has failed, but all retries have been exhausted.
            \end{tabular} \hypertarget{pv:enbl-led}{}\\ \hline
        % --- row ---
        EnblLED-Sel & Enum: No (0), Yes (1) & \begin{tabular}{@{}m{6cm}@{}}
                Enable LED. The voltage applied to the LED output depends on the power level, LED threshold, and power scale factor parameters.
            \end{tabular} \hypertarget{}{}\\ \hline
        % --- row ---
        EnblLED-Sts & Enum: No (0), Yes (1) & \begin{tabular}{@{}m{6cm}@{}}
                Status of the LED, i.e., if the LED is enabled or not.
            \end{tabular} \hypertarget{pv:led-pwr-lvl}{}\\ \hline
        % --- row ---
        LEDPwrLvl-SP & Float: min 0, max 100 & \begin{tabular}{@{}m{6cm}@{}}
                LED power level, specified as a percentage of maximum brightness.
            \end{tabular} \hypertarget{}{}\\ \hline
        % --- row ---
        LEDPwrLvl-RB & Float & \begin{tabular}{@{}m{6cm}@{}}
                The readback value of the LED power, as a percentage of maximum brightness.
            \end{tabular} \hypertarget{pv:led-thold}{}\\ \hline
        % --- row ---
        LEDThold-SP & Float: min 0, max 10 & \begin{tabular}{@{}m{6cm}@{}}
                LED threshold level. This is the minimum voltage necessary for the LED to emit (a perceptible amount of) light.
            \end{tabular} \hypertarget{}{}\\ \hline
        % --- row ---
        LEDThold-RB & Float & \begin{tabular}{@{}m{6cm}@{}}
                The readback value of the LED threshold value, i.e., the minimum voltage necessary for the LED to emit light.
            \end{tabular} \hypertarget{pv:led-pwr-scale-factor}{}\\ \hline
        % --- row ---
        LEDPwrScaleFactor-SP & Float: min 1e-6, max 0.02 & \begin{tabular}{@{}m{6cm}@{}}
                The LED power scale factor. This factor controls the maximum voltage level that can be applied to the LED terminals. The formula used to calculate voltage output to the LEDs is $A.e^{B.C}$, where A = LED voltage threshold, B = LED power level (0 to 100), and C = LED power scale factor.
            \end{tabular} \hypertarget{}{}\\ \hline
        % --- row ---
        LEDPwrScaleFactor-RB & Float & \begin{tabular}{@{}m{6cm}@{}}
                The readback value of the LED power scale factor.
            \end{tabular} \hypertarget{}{}\\ \hline
    \end{longtable}

\end{document}
\grid

